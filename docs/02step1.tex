\subsection{Simple matrix-matrix multiplication}

In our discussions, we will consider the computation
$$C:=AB + C$$
where $A$, $B$, and $C$ are $m\times k$, $k\times n$, $m\times n$  matrices.
respectively. 
Letting 
{\footnotesize%
\[
A = 
\left( \begin{array}{c c c c}
\alpha_{0,0} & \cdots & \alpha_{0,k-1} \\
\vdots &  & \vdots \\
\alpha_{m-1,0} & \cdots & \alpha_{m-1,k-1} \\
\end{array}
\right),
B = 
\left( \begin{array}{c c c c}
\beta_{0,0} & \cdots & \beta_{0,n-1} \\
\vdots &  & \vdots \\
\beta_{k-1,0} & \cdots & \beta_{k-1,n-1} \\
\end{array}
\right), \mbox{~and~}
C = 
\left( \begin{array}{c c c c}
\gamma_{0,0} & \cdots & \gamma_{0,n-1} \\
\vdots &  & \vdots \\
\gamma_{m-1,0} & \cdots & \gamma_{m-1,n-1} \\
\end{array}
\right).
\]%
}
$ C := A B + C $ computes
\[
\gamma_{i,j} := \sum_{p=0}^{k-1} \alpha_{i,p} \beta_{p,j} +
\gamma_{i,j}.
\]
If $ A $, $ B $, and $ C $ are stored as floating point numbers in two-dimensional arrays {\tt
  A}, {\tt B}, and {\tt C}, 
the following pseudocode computes $ C := A B + C $:
%\begin{algorithm}[!htp]
%\begin{algorithmic}
%  \FOR {$i=0:1:m-1$}
%    \FOR {$j=0:1:n-1$}
%      \FOR {$p=0:1:k-1$}
%        \STATE {$C(i,j) += C(i,j) + A(i,k)*B(k,j)$;}
%      \ENDFOR
%    \ENDFOR
%  \ENDFOR
%\end{algorithmic}
%\caption{Naive \GEMM{} Approach} 
%\label{a:3loops}
%\end{algorithm}
%%% \begin{tabular}{l@{}}
%%% {\bf for} $j\!=\! 0:n\!-\!1$ {\bf steps of} $1$ \\
%%%  \hspace{2ex}  {\bf for} $p \!=\! 0:k\!-\!1$ {\bf steps of} $1$ \\
%%%  \hspace{4ex}           {\bf for} $i \!=\! 0:m\!-\!1$ {\bf steps of} $1$ \\
%%% \hspace{6ex}             \textcolor{black}{$C(i,j)$}~$\mathrel{\!+\!}=$~\textcolor{black}{$A(i,p)$}~\textcolor{black}{$B(p,j)$} \\
%%% \hspace{4ex} {\bf endfor}\\
%%% \hspace{2ex} {\bf endfor}\\ 
%%% {\bf endfor}\\ 
%%% \end{tabular}

\vspace{0.1in}
\begin{center}
\begin{minipage}{3in}
\begin{verbatim}
for i=0:m-1
   for j=0:n-1
      for p=0:k-1
         C( i,j ) := A( i,p ) * B( p,j ) + C( i,j )
      endfor
   endfor
endfor
\end{verbatim}
\end{minipage}
\end{center}
\vspace{0.1in}
Counting a multiply and an add separately,
the computation requires $ 2 m n k $  floating point operations flops.

\subsection{Set up}

To let you efficiently learn about how to efficiently compute, you
start your project with much of the infrastructure in place.  We have
structured the subdirectory, {\tt step1}, somewhat like a project that
implements a real library might.   This may be overkill for our
purposes, but how to structure a software project is a useful skill to learn.

\begin{figure}[tb!]
	\begin{center}
\begin{minipage}{4in}
	\dirtree{%
		.1 step1.
		.2 README 
%%%\DTcomment{
%%%  \rm \color{red}
%%%  contents of the directory, how to compile and execute the source code{.} 
%%%}
.
		.2 sourceme.sh 
%%%\DTcomment{
%%%			\rm \color{red}
%%%			comment{.} 
%%%		}
.
		.2 makefile 
%%% \DTcomment{
%%%			\rm \color{red}
%%%			comment{.} 
%%%		}
.
		.2 dgemm 
%%%\DTcomment{
%%%			\rm \color{red}
%%%			comment{.} 
%%%		}
.
		.3 bl\_dgemm.c 
%%% \DTcomment{
%%%			\rm \color{red}
%%%			comment{.} 
%%%		}
.
		.3 bl\_dgemm\_ref.c 
%%%\DTcomment{
%%%			\rm \color{red}
%%%			comment{.} 
%%%		}
.
		.3 bl\_dgemm\_util.c 
%%% \DTcomment{
%%%			\rm \color{red}
%%%			comment{.} 
%%%		}
.
		.2 include 
%%%\DTcomment{
%%%			\rm \color{red}
%%%			comment{.} 
%%%		}
.
		.3 bl\_dgemm.h 
%%% \DTcomment{
%%%			\rm \color{red}
%%%			comment{.} 
%%%		}
.
		.3 bl\_dgemm\_ref.h 
%%%\DTcomment{
%%%			\rm \color{red}
%%%			comment{.} 
%%%		}
.
		.3 bl\_config.h 
%%% \DTcomment{
%%%			\rm \color{red}
%%%			comment{.} 
%%%		}
.
		.2 lib 
%%%\DTcomment{
%%%			\rm \color{red}
%%%			comment{.} 
%%%		}
.
		.2 makefile.in.files
%%% \DTcomment{
%%%			\rm \color{red}
%%%			comment{.} 
%%%		}
.
		.3 make.intel.inc 
%%% \DTcomment{
%%%			\rm \color{red}
%%%			comment{.} 
%%%		}
.
		.3 make.gnu.inc 
%%%\DTcomment{
%%%			\rm \color{red}
%%%			comment{.} 
%%%		}
.
		.3 make.inc 
%%% \DTcomment{
%%%			\rm \color{red}
%%%			comment{.} 
%%%		}
.
		.2 test 
%%%\DTcomment{
%%%			\rm \color{red}
%%%			comment{.} 
%%%		}
.
		.3 makefile 
%%% \DTcomment{
%%%			\rm \color{red}
%%%			comment{.} 
%%%		}
.
		.3 test\_bl\_dgemm.c 
%%%\DTcomment{
%%%			\rm \color{red}
%%%			comment{.} 
%%%		}
.
		.3 run\_bl\_dgemm.sh 
%%% \DTcomment{
%%%			\rm \color{red}
%%%			comment{.} 
%%%		}
.
		.3 test\_bl\_dgemm.x 
%%%\DTcomment{
%%%			\rm \color{red}
%%%			comment{.} 
%%%		}
.       
		.3 tacc\_run\_bl\_dgemm.sh 
%%% \DTcomment{
%%%			\rm \color{red}
%%%			comment{.} 
%%%		}
.	
	}
\end{minipage}
\end{center}
\caption{Structure of directory {\tt step1}.}
\label{fig:DirStep1}
\end{figure}

Consider Figure~\ref{fig:DirStep1}, which illustrates the directory
structure for subdirectory {\tt step1}:
\begin{description}
\item[{\tt README}]
Is a file that describes the contents of the directory.
\item[{\tt dgemm}]
Is the subdirectory routines that implement \Gemm\ can be found.
In it
\begin{description}
\item[{\tt bl\_dgemm\_ref}] contains the routine {\tt dgemm\_ref} that
  is a simple implementation of \Gemm\ that you will use to check the
  correctness of your implementations.
\item[{\tt my\_dgemm}] contains the routine {\tt dgemm} that
  that initially is a simple implementation of \Gemm\ and that you will optimize as part of the first step on your way to mastering how to optimize \Gemm.
\item[{\tt bl\_dgemm\_util}] contains utility routines that will come in handy later.
\end{description}
\item[{\tt include}]
This directory contains 
include files with various macro definitions and 
other header information.
\item[{\tt lib}] This directory will hold libraries that you will install so you can compare your performance with existing high-performance solutions.
\item[{\tt test}] This directory contains "test drivers" for the various implementations.
\item[{\tt }]
\end{description}


