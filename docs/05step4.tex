The benefit of the BLIS way of structuring the GotoBLAS approach to the implementation of \Gemm\ is that it exposes five loops in {tt C} which can then be easily parallelized with OpenMP directives.

\subsection{To parallelize or not to parallelize, that's the question}

The fundamental question becomes which loop to parallelize.  In~\cite{BLIS3}
\begin{quote}
	Tyler M. Smith, Robert van de Geijn, Mikhail  Smelyanskiy, Jeff R. Hammond, and Field G. Van Zee.
	\myhref{http://ieeexplore.ieee.org/xpl/articleDetails.jsp?arnumber=6877334}{Anatomy of High-Performance Many-Threaded Matrix Multiplication.} IEEE 28th International Parallel and Distributed Processing Symposium, 2014.  Also available from \myhref{http://shpc.ices.utexas.edu/publications.html}{http://shpc.ices.utexas.edu/publications.html}.
\end{quote} 
a detailed discussion is given of what the pros and cons are regarding the parallelization of each loop.
For multi-core architectures (multi-threaded architectures with relatively few cores) results can be found in the earlier paper~\cite{BLIS2} 
\begin{quote}
	Field G. Van Zee, Tyler Smith, Bryan Marker, Tze Meng Low, Robert A. van de Geijn, Francisco D. Igual, Mikhail Smelyanskiy, Xianyi Zhang, Michael Kistler, Vernon Austel, John Gunnels, Lee Killough. 
	\myhref{}{The BLIS Framework: Experiments in Portability.}		
		ACM Transactions on Mathematical Software.  To appear. Also available from \myhref{http://shpc.ices.utexas.edu/publications.html}{http://shpc.ices.utexas.edu/publications.html}.
\end{quote}
